% !TEX TS-program = xelatex
% !TEX encoding = UTF-8 Unicode
% !Mode:: "TeX:UTF-8"

\documentclass{resume}
\usepackage{zh_CN-Adobefonts_external} % Simplified Chinese Support using external fonts (./fonts/zh_CN-Adobe/)
%\usepackage{zh_CN-Adobefonts_internal} % Simplified Chinese Support using system fonts
\usepackage{linespacing_fix} % disable extra space before next section
\usepackage{cite}

\begin{document}
\pagenumbering{gobble} % suppress displaying page number

\name{卢威}

% {E-mail}{mobilephone}{homepage}
% be careful of _ in emaill address
\contactInfo{\faMobile :+86 13419656478}{\faEnvelope :whuluwei@gmail.com}{\faLink :knoway.xyz}{\faGithub :luwei14}
% {E-mail}{mobilephone}
% keep the last empty braces!
%\contactInfo{xxx@yuanbin.me}{(+86) 131-221-87xxx}{}
 
\section{个人简介}
我是卢威。我分别在2012年和2021年从武汉大学获得GIS领域的理学学士学位和工学博士学位。我的学术背景主要在地图学和地理信息科学领域,尤其是几何算法和地理可视化。当前,我主要专注于面向地理空间数据的拓扑数据分析(TDA)理论和方法的研究,即利用代数拓扑的理论方法研究空间大数据问题。个人爱好阅读、跑步和骑行。曾参加过三次马拉松比赛,并完赛,最好成绩4小时11分。曾在2020年暑假期间,历时60日,完成沿胡焕庸线的长途骑行,从云南腾冲到黑龙江黑河,全程5800余公里。

\section{教育背景}
\datedsubsection{\textbf{武汉大学},地图制图学与地理信息工程,\textit{工学博士}}{2021年6月}
\datedsubsection{\textbf{武汉大学},地理信息系统,\textit{理学学士}}{2012年6月}

\section{发表论文}
Google Scholar, Citations: 90, h-index: 3.
\begin{itemize}[parsep=1.0ex]
\item[1.] \textbf{卢威}, 艾廷华. (2020). 利用三角剖分骨架图提取简单多边形目标中心点[J]. 武汉大学学报·信息科学版, 45(3) 337-343. 

\item[2.] 王维才, 艾廷华, 晏雄锋, \textbf{卢巍}. (2020). 多约束条件下的正六边形格网室内路径规划[J]. 武汉大学学报·信息科学版, 45(1), 111-118.

\item[3.] 郑建滨, 艾廷华, 晏雄锋, \textbf{卢威}. (2019). 基于XGBoost的多建筑WiFi位置指纹室内定位方法[J]. 测绘地理信息, 44(2), 65-68.

\item[4.] \textbf{Wei Lu}, Wei Yang, and Tinghua Ai. (2018) Evaluating Non-Motorized Transport Popularity of Urban Roads by Sports GPS Tracks[C]. In 2018 26th International Conference on Geoinformatics. IEEE. 

\item[5.] Wei Yang, and Tinghua Ai, \textbf{Wei Lu}, Tong Zhang. (2018) Cycle Periodic Behavior Detection and Sports Place Extraction Using Crowdsourced Running Trace Data[C]. In 2018 26th International Conference on Geoinformatics. IEEE. 

\item[6.] Wei Yang, Tinghua Ai, \textbf{Wei Lu}. (2018) A Method for Extracting Road Boundary Information from Crowdsourcing Vehicle GPS Trajectories[J]. Sensors, 18(4), 1261.

\item[7.] \textbf{Wei Lu} and Tinghua Ai. (2018) Center Point of Simple Area Feature Based on Triangulation Skeleton Graph[C]. Leibniz International Proceedings in Informatics (LIPIcs). VOL. 114, 41:1- 41:6. 10th International Conference on Geographic Information Science (GIScience 2018). 

\item[8.] \textbf{Wei Lu}, Tinghua Ai, Xiang Zhang and Yakun He. (2017). An Interactive Web Mapping Visualization of Urban Air Quality Monitoring Data of China[J]. Atmosphere, 8(8), 148. 

\item[9.] 杨敏, 艾廷华, \textbf{卢威}, 成晓强, 周启. (2015). 自发地理信息兴趣点数据在线综与多尺度可视化方法[J]. 测绘学报, 44(2), 228-234. 
\end{itemize}

\section{授权专利}
\begin{itemize}[parsep=1.0ex]
  \item[1.] 陈小祥, \textbf{卢威}, 司马晓等. 图斑简化方法、装置、设备和计算机可读存储介质. 专利号: ZL202010414866.2, 申请日: 2020.05.15, 授权日: 2021.07.06, 中国.
\end{itemize}

\section{项目信息}
\datedsubsection{\textbf{武汉大学}}{2012.6-2020.9}
\begin{itemize}[parsep=1.0ex]
  \item[1.] 2012-2014, 863子课题, 大规模复杂地理空间数据可视化技术与自适应制图技术研究, 项目骨干
  \item[2.] 2016-2020, 国家自然科学基金重点项目, 网络众源地理信息在线式尺度变换原理与方法, 参与
\end{itemize}

\datedsubsection{\textbf{江门市勘测院}}{2018.8-2020.12}
\begin{itemize}[parsep=1.0ex]
  \item[1.] 2018.8-2019.2, 江门市勘测院房屋规划条件核实测绘软件, 负责人
  \item[2.] 2020.6-2020.12, 江门市勘测院道路规划条件核实测绘软件, 负责人
\end{itemize}

\datedsubsection{\textbf{深圳市城市规划设计研究院}}{2019.8-2020.5}
\begin{itemize}[parsep=1.0ex]
  \item[1.] 2019.8-2020.1, 国土空间规划双评价系统, 负责人
  \item[2.] 2020.1-2020.5, 玉塘街道空间信息平台, 负责人
\end{itemize}

\section{软件登记}
\begin{itemize}[parsep=1.0ex]
  \item[1.] 泛在网络空间数据在线自适应制图系统,2015SR018131
  \item[2.] 泛在空间POI可视化与制图系统,2014R11L013994
  \item[3.] 分布式国土空间规划双评价分析系统 2020SR0242117
\end{itemize}

\section{荣誉获奖}
\begin{itemize}[parsep=1.0ex]
  \item[1.] 武汉大学空间信息协同创新中心研究生奖学金, 2015年
  \item[2.] 武汉大学研究生新生奖学金, 2012年
  \item[3.] 武汉大学优秀毕业生, 2012年
  \item[4.] 2010年全国大学生数学建模竞赛, 国家二等奖, 卢威,刘嘉玮,张淑祥, 2010年
  \item[5.] 第一届中国大学生数学竞赛, 湖北赛区三等奖, 2009年
  \item[6.] 国家励志奖学金, 2009年
\end{itemize}

\section{辅助指导}
\begin{itemize}[parsep=1.0ex]
  \item[1.] Faisal Shehzad, \textit{Master Thesis}, \textbf{Automated Generalization of Roads and Buildings in Urban Map Representation}, 2018
\end{itemize}
\end{document}
